\documentclass[11pt]{article}

    \usepackage[breakable]{tcolorbox}
    \usepackage{parskip} % Stop auto-indenting (to mimic markdown behaviour)
    
    \usepackage{iftex}
    \ifPDFTeX
    	\usepackage[T1]{fontenc}
    	\usepackage{mathpazo}
    \else
    	\usepackage{fontspec}
    \fi

    % Basic figure setup, for now with no caption control since it's done
    % automatically by Pandoc (which extracts ![](path) syntax from Markdown).
    \usepackage{graphicx}
    % Maintain compatibility with old templates. Remove in nbconvert 6.0
   \let\Oldincludegraphics\includegraphics
    % Ensure that by default, figures have no caption (until we provide a
    % proper Figure object with a Caption API and a way to capture that
    % in the conversion process - todo).
    \usepackage{caption}
    \DeclareCaptionFormat{nocaption}{}
    \captionsetup{format=nocaption,aboveskip=0pt,belowskip=0pt}

    \usepackage[Export]{adjustbox} % Used to constrain images to a maximum size
    \adjustboxset{max size={0.9\linewidth}{0.9\paperheight}}
    \usepackage{float}
    \floatplacement{figure}{H} % forces figures to be placed at the correct location
    \usepackage{xcolor} % Allow colors to be defined
    \usepackage{enumerate} % Needed for markdown enumerations to work
    \usepackage{geometry} % Used to adjust the document margins
    \usepackage{amsmath} % Equations
    \usepackage{amssymb} % Equations
    \usepackage{textcomp} % defines textquotesingle
    % Hack from http://tex.stackexchange.com/a/47451/13684:
    \AtBeginDocument{%
        \def\PYZsq{\textquotesingle}% Upright quotes in Pygmentized code
    }
    \usepackage{upquote} % Upright quotes for verbatim code
    \usepackage{eurosym} % defines \euro
    \usepackage[mathletters]{ucs} % Extended unicode (utf-8) support
    \usepackage{fancyvrb} % verbatim replacement that allows latex
    \usepackage{grffile} % extends the file name processing of package graphics 
                         % to support a larger range
    \makeatletter % fix for grffile with XeLaTeX
    \def\Gread@@xetex#1{%
      \IfFileExists{"\Gin@base".bb}%
      {\Gread@eps{\Gin@base.bb}}%
      {\Gread@@xetex@aux#1}%
    }
    \makeatother

    % The hyperref package gives us a pdf with properly built
    % internal navigation ('pdf bookmarks' for the table of contents,
    % internal cross-reference links, web links for URLs, etc.)
    \usepackage{hyperref}
    % The default LaTeX title has an obnoxious amount of whitespace. By default,
    % titling removes some of it. It also provides customization options.
    \usepackage{titling}
    \usepackage{longtable} % longtable support required by pandoc >1.10
    \usepackage{booktabs}  % table support for pandoc > 1.12.2
    \usepackage[inline]{enumitem} % IRkernel/repr support (it uses the enumerate* environment)
    \usepackage[normalem]{ulem} % ulem is needed to support strikethroughs (\sout)
                                % normalem makes italics be italics, not underlines
    \usepackage{mathrsfs}
    

    
    % Colors for the hyperref package
    \definecolor{urlcolor}{rgb}{0,.145,.698}
    \definecolor{linkcolor}{rgb}{.71,0.21,0.01}
    \definecolor{citecolor}{rgb}{.12,.54,.11}

    % ANSI colors
    \definecolor{ansi-black}{HTML}{3E424D}
    \definecolor{ansi-black-intense}{HTML}{282C36}
    \definecolor{ansi-red}{HTML}{E75C58}
    \definecolor{ansi-red-intense}{HTML}{B22B31}
    \definecolor{ansi-green}{HTML}{00A250}
    \definecolor{ansi-green-intense}{HTML}{007427}
    \definecolor{ansi-yellow}{HTML}{DDB62B}
    \definecolor{ansi-yellow-intense}{HTML}{B27D12}
    \definecolor{ansi-blue}{HTML}{208FFB}
    \definecolor{ansi-blue-intense}{HTML}{0065CA}
    \definecolor{ansi-magenta}{HTML}{D160C4}
    \definecolor{ansi-magenta-intense}{HTML}{A03196}
    \definecolor{ansi-cyan}{HTML}{60C6C8}
    \definecolor{ansi-cyan-intense}{HTML}{258F8F}
    \definecolor{ansi-white}{HTML}{C5C1B4}
    \definecolor{ansi-white-intense}{HTML}{A1A6B2}
    \definecolor{ansi-default-inverse-fg}{HTML}{FFFFFF}
    \definecolor{ansi-default-inverse-bg}{HTML}{000000}

    % commands and environments needed by pandoc snippets
    % extracted from the output of `pandoc -s`
    \providecommand{\tightlist}{%
      \setlength{\itemsep}{0pt}\setlength{\parskip}{0pt}}
    \DefineVerbatimEnvironment{Highlighting}{Verbatim}{commandchars=\\\{\}}
    % Add ',fontsize=\small' for more characters per line
    \newenvironment{Shaded}{}{}
    \newcommand{\KeywordTok}[1]{\textcolor[rgb]{0.00,0.44,0.13}{\textbf{{#1}}}}
    \newcommand{\DataTypeTok}[1]{\textcolor[rgb]{0.56,0.13,0.00}{{#1}}}
    \newcommand{\DecValTok}[1]{\textcolor[rgb]{0.25,0.63,0.44}{{#1}}}
    \newcommand{\BaseNTok}[1]{\textcolor[rgb]{0.25,0.63,0.44}{{#1}}}
    \newcommand{\FloatTok}[1]{\textcolor[rgb]{0.25,0.63,0.44}{{#1}}}
    \newcommand{\CharTok}[1]{\textcolor[rgb]{0.25,0.44,0.63}{{#1}}}
    \newcommand{\StringTok}[1]{\textcolor[rgb]{0.25,0.44,0.63}{{#1}}}
    \newcommand{\CommentTok}[1]{\textcolor[rgb]{0.38,0.63,0.69}{\textit{{#1}}}}
    \newcommand{\OtherTok}[1]{\textcolor[rgb]{0.00,0.44,0.13}{{#1}}}
    \newcommand{\AlertTok}[1]{\textcolor[rgb]{1.00,0.00,0.00}{\textbf{{#1}}}}
    \newcommand{\FunctionTok}[1]{\textcolor[rgb]{0.02,0.16,0.49}{{#1}}}
    \newcommand{\RegionMarkerTok}[1]{{#1}}
    \newcommand{\ErrorTok}[1]{\textcolor[rgb]{1.00,0.00,0.00}{\textbf{{#1}}}}
    \newcommand{\NormalTok}[1]{{#1}}
    
    % Additional commands for more recent versions of Pandoc
    \newcommand{\ConstantTok}[1]{\textcolor[rgb]{0.53,0.00,0.00}{{#1}}}
    \newcommand{\SpecialCharTok}[1]{\textcolor[rgb]{0.25,0.44,0.63}{{#1}}}
    \newcommand{\VerbatimStringTok}[1]{\textcolor[rgb]{0.25,0.44,0.63}{{#1}}}
    \newcommand{\SpecialStringTok}[1]{\textcolor[rgb]{0.73,0.40,0.53}{{#1}}}
    \newcommand{\ImportTok}[1]{{#1}}
    \newcommand{\DocumentationTok}[1]{\textcolor[rgb]{0.73,0.13,0.13}{\textit{{#1}}}}
    \newcommand{\AnnotationTok}[1]{\textcolor[rgb]{0.38,0.63,0.69}{\textbf{\textit{{#1}}}}}
    \newcommand{\CommentVarTok}[1]{\textcolor[rgb]{0.38,0.63,0.69}{\textbf{\textit{{#1}}}}}
    \newcommand{\VariableTok}[1]{\textcolor[rgb]{0.10,0.09,0.49}{{#1}}}
    \newcommand{\ControlFlowTok}[1]{\textcolor[rgb]{0.00,0.44,0.13}{\textbf{{#1}}}}
    \newcommand{\OperatorTok}[1]{\textcolor[rgb]{0.40,0.40,0.40}{{#1}}}
    \newcommand{\BuiltInTok}[1]{{#1}}
    \newcommand{\ExtensionTok}[1]{{#1}}
    \newcommand{\PreprocessorTok}[1]{\textcolor[rgb]{0.74,0.48,0.00}{{#1}}}
    \newcommand{\AttributeTok}[1]{\textcolor[rgb]{0.49,0.56,0.16}{{#1}}}
    \newcommand{\InformationTok}[1]{\textcolor[rgb]{0.38,0.63,0.69}{\textbf{\textit{{#1}}}}}
    \newcommand{\WarningTok}[1]{\textcolor[rgb]{0.38,0.63,0.69}{\textbf{\textit{{#1}}}}}
    
    
    % Define a nice break command that doesn't care if a line doesn't already
    % exist.
    \def\br{\hspace*{\fill} \\* }
    % Math Jax compatibility definitions
    \def\gt{>}
    \def\lt{<}
    \let\Oldtex\TeX
    \let\Oldlatex\LaTeX
    \renewcommand{\TeX}{\textrm{\Oldtex}}
    \renewcommand{\LaTeX}{\textrm{\Oldlatex}}
    % Document parameters
    
    
    
% Pygments definitions
\makeatletter
\def\PY@reset{\let\PY@it=\relax \let\PY@bf=\relax%
    \let\PY@ul=\relax \let\PY@tc=\relax%
    \let\PY@bc=\relax \let\PY@ff=\relax}
\def\PY@tok#1{\csname PY@tok@#1\endcsname}
\def\PY@toks#1+{\ifx\relax#1\empty\else%
    \PY@tok{#1}\expandafter\PY@toks\fi}
\def\PY@do#1{\PY@bc{\PY@tc{\PY@ul{%
    \PY@it{\PY@bf{\PY@ff{#1}}}}}}}
\def\PY#1#2{\PY@reset\PY@toks#1+\relax+\PY@do{#2}}

\expandafter\def\csname PY@tok@w\endcsname{\def\PY@tc##1{\textcolor[rgb]{0.73,0.73,0.73}{##1}}}
\expandafter\def\csname PY@tok@c\endcsname{\let\PY@it=\textit\def\PY@tc##1{\textcolor[rgb]{0.25,0.50,0.50}{##1}}}
\expandafter\def\csname PY@tok@cp\endcsname{\def\PY@tc##1{\textcolor[rgb]{0.74,0.48,0.00}{##1}}}
\expandafter\def\csname PY@tok@k\endcsname{\let\PY@bf=\textbf\def\PY@tc##1{\textcolor[rgb]{0.00,0.50,0.00}{##1}}}
\expandafter\def\csname PY@tok@kp\endcsname{\def\PY@tc##1{\textcolor[rgb]{0.00,0.50,0.00}{##1}}}
\expandafter\def\csname PY@tok@kt\endcsname{\def\PY@tc##1{\textcolor[rgb]{0.69,0.00,0.25}{##1}}}
\expandafter\def\csname PY@tok@o\endcsname{\def\PY@tc##1{\textcolor[rgb]{0.40,0.40,0.40}{##1}}}
\expandafter\def\csname PY@tok@ow\endcsname{\let\PY@bf=\textbf\def\PY@tc##1{\textcolor[rgb]{0.67,0.13,1.00}{##1}}}
\expandafter\def\csname PY@tok@nb\endcsname{\def\PY@tc##1{\textcolor[rgb]{0.00,0.50,0.00}{##1}}}
\expandafter\def\csname PY@tok@nf\endcsname{\def\PY@tc##1{\textcolor[rgb]{0.00,0.00,1.00}{##1}}}
\expandafter\def\csname PY@tok@nc\endcsname{\let\PY@bf=\textbf\def\PY@tc##1{\textcolor[rgb]{0.00,0.00,1.00}{##1}}}
\expandafter\def\csname PY@tok@nn\endcsname{\let\PY@bf=\textbf\def\PY@tc##1{\textcolor[rgb]{0.00,0.00,1.00}{##1}}}
\expandafter\def\csname PY@tok@ne\endcsname{\let\PY@bf=\textbf\def\PY@tc##1{\textcolor[rgb]{0.82,0.25,0.23}{##1}}}
\expandafter\def\csname PY@tok@nv\endcsname{\def\PY@tc##1{\textcolor[rgb]{0.10,0.09,0.49}{##1}}}
\expandafter\def\csname PY@tok@no\endcsname{\def\PY@tc##1{\textcolor[rgb]{0.53,0.00,0.00}{##1}}}
\expandafter\def\csname PY@tok@nl\endcsname{\def\PY@tc##1{\textcolor[rgb]{0.63,0.63,0.00}{##1}}}
\expandafter\def\csname PY@tok@ni\endcsname{\let\PY@bf=\textbf\def\PY@tc##1{\textcolor[rgb]{0.60,0.60,0.60}{##1}}}
\expandafter\def\csname PY@tok@na\endcsname{\def\PY@tc##1{\textcolor[rgb]{0.49,0.56,0.16}{##1}}}
\expandafter\def\csname PY@tok@nt\endcsname{\let\PY@bf=\textbf\def\PY@tc##1{\textcolor[rgb]{0.00,0.50,0.00}{##1}}}
\expandafter\def\csname PY@tok@nd\endcsname{\def\PY@tc##1{\textcolor[rgb]{0.67,0.13,1.00}{##1}}}
\expandafter\def\csname PY@tok@s\endcsname{\def\PY@tc##1{\textcolor[rgb]{0.73,0.13,0.13}{##1}}}
\expandafter\def\csname PY@tok@sd\endcsname{\let\PY@it=\textit\def\PY@tc##1{\textcolor[rgb]{0.73,0.13,0.13}{##1}}}
\expandafter\def\csname PY@tok@si\endcsname{\let\PY@bf=\textbf\def\PY@tc##1{\textcolor[rgb]{0.73,0.40,0.53}{##1}}}
\expandafter\def\csname PY@tok@se\endcsname{\let\PY@bf=\textbf\def\PY@tc##1{\textcolor[rgb]{0.73,0.40,0.13}{##1}}}
\expandafter\def\csname PY@tok@sr\endcsname{\def\PY@tc##1{\textcolor[rgb]{0.73,0.40,0.53}{##1}}}
\expandafter\def\csname PY@tok@ss\endcsname{\def\PY@tc##1{\textcolor[rgb]{0.10,0.09,0.49}{##1}}}
\expandafter\def\csname PY@tok@sx\endcsname{\def\PY@tc##1{\textcolor[rgb]{0.00,0.50,0.00}{##1}}}
\expandafter\def\csname PY@tok@m\endcsname{\def\PY@tc##1{\textcolor[rgb]{0.40,0.40,0.40}{##1}}}
\expandafter\def\csname PY@tok@gh\endcsname{\let\PY@bf=\textbf\def\PY@tc##1{\textcolor[rgb]{0.00,0.00,0.50}{##1}}}
\expandafter\def\csname PY@tok@gu\endcsname{\let\PY@bf=\textbf\def\PY@tc##1{\textcolor[rgb]{0.50,0.00,0.50}{##1}}}
\expandafter\def\csname PY@tok@gd\endcsname{\def\PY@tc##1{\textcolor[rgb]{0.63,0.00,0.00}{##1}}}
\expandafter\def\csname PY@tok@gi\endcsname{\def\PY@tc##1{\textcolor[rgb]{0.00,0.63,0.00}{##1}}}
\expandafter\def\csname PY@tok@gr\endcsname{\def\PY@tc##1{\textcolor[rgb]{1.00,0.00,0.00}{##1}}}
\expandafter\def\csname PY@tok@ge\endcsname{\let\PY@it=\textit}
\expandafter\def\csname PY@tok@gs\endcsname{\let\PY@bf=\textbf}
\expandafter\def\csname PY@tok@gp\endcsname{\let\PY@bf=\textbf\def\PY@tc##1{\textcolor[rgb]{0.00,0.00,0.50}{##1}}}
\expandafter\def\csname PY@tok@go\endcsname{\def\PY@tc##1{\textcolor[rgb]{0.53,0.53,0.53}{##1}}}
\expandafter\def\csname PY@tok@gt\endcsname{\def\PY@tc##1{\textcolor[rgb]{0.00,0.27,0.87}{##1}}}
\expandafter\def\csname PY@tok@err\endcsname{\def\PY@bc##1{\setlength{\fboxsep}{0pt}\fcolorbox[rgb]{1.00,0.00,0.00}{1,1,1}{\strut ##1}}}
\expandafter\def\csname PY@tok@kc\endcsname{\let\PY@bf=\textbf\def\PY@tc##1{\textcolor[rgb]{0.00,0.50,0.00}{##1}}}
\expandafter\def\csname PY@tok@kd\endcsname{\let\PY@bf=\textbf\def\PY@tc##1{\textcolor[rgb]{0.00,0.50,0.00}{##1}}}
\expandafter\def\csname PY@tok@kn\endcsname{\let\PY@bf=\textbf\def\PY@tc##1{\textcolor[rgb]{0.00,0.50,0.00}{##1}}}
\expandafter\def\csname PY@tok@kr\endcsname{\let\PY@bf=\textbf\def\PY@tc##1{\textcolor[rgb]{0.00,0.50,0.00}{##1}}}
\expandafter\def\csname PY@tok@bp\endcsname{\def\PY@tc##1{\textcolor[rgb]{0.00,0.50,0.00}{##1}}}
\expandafter\def\csname PY@tok@fm\endcsname{\def\PY@tc##1{\textcolor[rgb]{0.00,0.00,1.00}{##1}}}
\expandafter\def\csname PY@tok@vc\endcsname{\def\PY@tc##1{\textcolor[rgb]{0.10,0.09,0.49}{##1}}}
\expandafter\def\csname PY@tok@vg\endcsname{\def\PY@tc##1{\textcolor[rgb]{0.10,0.09,0.49}{##1}}}
\expandafter\def\csname PY@tok@vi\endcsname{\def\PY@tc##1{\textcolor[rgb]{0.10,0.09,0.49}{##1}}}
\expandafter\def\csname PY@tok@vm\endcsname{\def\PY@tc##1{\textcolor[rgb]{0.10,0.09,0.49}{##1}}}
\expandafter\def\csname PY@tok@sa\endcsname{\def\PY@tc##1{\textcolor[rgb]{0.73,0.13,0.13}{##1}}}
\expandafter\def\csname PY@tok@sb\endcsname{\def\PY@tc##1{\textcolor[rgb]{0.73,0.13,0.13}{##1}}}
\expandafter\def\csname PY@tok@sc\endcsname{\def\PY@tc##1{\textcolor[rgb]{0.73,0.13,0.13}{##1}}}
\expandafter\def\csname PY@tok@dl\endcsname{\def\PY@tc##1{\textcolor[rgb]{0.73,0.13,0.13}{##1}}}
\expandafter\def\csname PY@tok@s2\endcsname{\def\PY@tc##1{\textcolor[rgb]{0.73,0.13,0.13}{##1}}}
\expandafter\def\csname PY@tok@sh\endcsname{\def\PY@tc##1{\textcolor[rgb]{0.73,0.13,0.13}{##1}}}
\expandafter\def\csname PY@tok@s1\endcsname{\def\PY@tc##1{\textcolor[rgb]{0.73,0.13,0.13}{##1}}}
\expandafter\def\csname PY@tok@mb\endcsname{\def\PY@tc##1{\textcolor[rgb]{0.40,0.40,0.40}{##1}}}
\expandafter\def\csname PY@tok@mf\endcsname{\def\PY@tc##1{\textcolor[rgb]{0.40,0.40,0.40}{##1}}}
\expandafter\def\csname PY@tok@mh\endcsname{\def\PY@tc##1{\textcolor[rgb]{0.40,0.40,0.40}{##1}}}
\expandafter\def\csname PY@tok@mi\endcsname{\def\PY@tc##1{\textcolor[rgb]{0.40,0.40,0.40}{##1}}}
\expandafter\def\csname PY@tok@il\endcsname{\def\PY@tc##1{\textcolor[rgb]{0.40,0.40,0.40}{##1}}}
\expandafter\def\csname PY@tok@mo\endcsname{\def\PY@tc##1{\textcolor[rgb]{0.40,0.40,0.40}{##1}}}
\expandafter\def\csname PY@tok@ch\endcsname{\let\PY@it=\textit\def\PY@tc##1{\textcolor[rgb]{0.25,0.50,0.50}{##1}}}
\expandafter\def\csname PY@tok@cm\endcsname{\let\PY@it=\textit\def\PY@tc##1{\textcolor[rgb]{0.25,0.50,0.50}{##1}}}
\expandafter\def\csname PY@tok@cpf\endcsname{\let\PY@it=\textit\def\PY@tc##1{\textcolor[rgb]{0.25,0.50,0.50}{##1}}}
\expandafter\def\csname PY@tok@c1\endcsname{\let\PY@it=\textit\def\PY@tc##1{\textcolor[rgb]{0.25,0.50,0.50}{##1}}}
\expandafter\def\csname PY@tok@cs\endcsname{\let\PY@it=\textit\def\PY@tc##1{\textcolor[rgb]{0.25,0.50,0.50}{##1}}}

\def\PYZbs{\char`\\}
\def\PYZus{\char`\_}
\def\PYZob{\char`\{}
\def\PYZcb{\char`\}}
\def\PYZca{\char`\^}
\def\PYZam{\char`\&}
\def\PYZlt{\char`\<}
\def\PYZgt{\char`\>}
\def\PYZsh{\char`\#}
\def\PYZpc{\char`\%}
\def\PYZdl{\char`\$}
\def\PYZhy{\char`\-}
\def\PYZsq{\char`\'}
\def\PYZdq{\char`\"}
\def\PYZti{\char`\~}
% for compatibility with earlier versions
\def\PYZat{@}
\def\PYZlb{[}
\def\PYZrb{]}
\makeatother


    % For linebreaks inside Verbatim environment from package fancyvrb. 
    \makeatletter
        \newbox\Wrappedcontinuationbox 
        \newbox\Wrappedvisiblespacebox 
        \newcommand*\Wrappedvisiblespace {\textcolor{red}{\textvisiblespace}} 
        \newcommand*\Wrappedcontinuationsymbol {\textcolor{red}{\llap{\tiny$\m@th\hookrightarrow$}}} 
        \newcommand*\Wrappedcontinuationindent {3ex } 
        \newcommand*\Wrappedafterbreak {\kern\Wrappedcontinuationindent\copy\Wrappedcontinuationbox} 
        % Take advantage of the already applied Pygments mark-up to insert 
        % potential linebreaks for TeX processing. 
        %        {, <, #, %, $, ' and ": go to next line. 
        %        _, }, ^, &, >, - and ~: stay at end of broken line. 
        % Use of \textquotesingle for straight quote. 
        \newcommand*\Wrappedbreaksatspecials {% 
            \def\PYGZus{\discretionary{\char`\_}{\Wrappedafterbreak}{\char`\_}}% 
            \def\PYGZob{\discretionary{}{\Wrappedafterbreak\char`\{}{\char`\{}}% 
            \def\PYGZcb{\discretionary{\char`\}}{\Wrappedafterbreak}{\char`\}}}% 
            \def\PYGZca{\discretionary{\char`\^}{\Wrappedafterbreak}{\char`\^}}% 
            \def\PYGZam{\discretionary{\char`\&}{\Wrappedafterbreak}{\char`\&}}% 
            \def\PYGZlt{\discretionary{}{\Wrappedafterbreak\char`\<}{\char`\<}}% 
            \def\PYGZgt{\discretionary{\char`\>}{\Wrappedafterbreak}{\char`\>}}% 
            \def\PYGZsh{\discretionary{}{\Wrappedafterbreak\char`\#}{\char`\#}}% 
            \def\PYGZpc{\discretionary{}{\Wrappedafterbreak\char`\%}{\char`\%}}% 
            \def\PYGZdl{\discretionary{}{\Wrappedafterbreak\char`\$}{\char`\$}}% 
            \def\PYGZhy{\discretionary{\char`\-}{\Wrappedafterbreak}{\char`\-}}% 
            \def\PYGZsq{\discretionary{}{\Wrappedafterbreak\textquotesingle}{\textquotesingle}}% 
            \def\PYGZdq{\discretionary{}{\Wrappedafterbreak\char`\"}{\char`\"}}% 
            \def\PYGZti{\discretionary{\char`\~}{\Wrappedafterbreak}{\char`\~}}% 
        } 
        % Some characters . , ; ? ! / are not pygmentized. 
        % This macro makes them "active" and they will insert potential linebreaks 
        \newcommand*\Wrappedbreaksatpunct {% 
            \lccode`\~`\.\lowercase{\def~}{\discretionary{\hbox{\char`\.}}{\Wrappedafterbreak}{\hbox{\char`\.}}}% 
            \lccode`\~`\,\lowercase{\def~}{\discretionary{\hbox{\char`\,}}{\Wrappedafterbreak}{\hbox{\char`\,}}}% 
            \lccode`\~`\;\lowercase{\def~}{\discretionary{\hbox{\char`\;}}{\Wrappedafterbreak}{\hbox{\char`\;}}}% 
            \lccode`\~`\:\lowercase{\def~}{\discretionary{\hbox{\char`\:}}{\Wrappedafterbreak}{\hbox{\char`\:}}}% 
            \lccode`\~`\?\lowercase{\def~}{\discretionary{\hbox{\char`\?}}{\Wrappedafterbreak}{\hbox{\char`\?}}}% 
            \lccode`\~`\!\lowercase{\def~}{\discretionary{\hbox{\char`\!}}{\Wrappedafterbreak}{\hbox{\char`\!}}}% 
            \lccode`\~`\/\lowercase{\def~}{\discretionary{\hbox{\char`\/}}{\Wrappedafterbreak}{\hbox{\char`\/}}}% 
            \catcode`\.\active
            \catcode`\,\active 
            \catcode`\;\active
            \catcode`\:\active
            \catcode`\?\active
            \catcode`\!\active
            \catcode`\/\active 
            \lccode`\~`\~ 	
        }
    \makeatother

    \let\OriginalVerbatim=\Verbatim
    \makeatletter
    \renewcommand{\Verbatim}[1][1]{%
        %\parskip\z@skip
        \sbox\Wrappedcontinuationbox {\Wrappedcontinuationsymbol}%
        \sbox\Wrappedvisiblespacebox {\FV@SetupFont\Wrappedvisiblespace}%
        \def\FancyVerbFormatLine ##1{\hsize\linewidth
            \vtop{\raggedright\hyphenpenalty\z@\exhyphenpenalty\z@
                \doublehyphendemerits\z@\finalhyphendemerits\z@
                \strut ##1\strut}%
        }%
        % If the linebreak is at a space, the latter will be displayed as visible
        % space at end of first line, and a continuation symbol starts next line.
        % Stretch/shrink are however usually zero for typewriter font.
        \def\FV@Space {%
            \nobreak\hskip\z@ plus\fontdimen3\font minus\fontdimen4\font
            \discretionary{\copy\Wrappedvisiblespacebox}{\Wrappedafterbreak}
            {\kern\fontdimen2\font}%
        }%
        
        % Allow breaks at special characters using \PYG... macros.
        \Wrappedbreaksatspecials
        % Breaks at punctuation characters . , ; ? ! and / need catcode=\active 	
        \OriginalVerbatim[#1,codes*=\Wrappedbreaksatpunct]%
    }
    \makeatother

    % Exact colors from NB
    \definecolor{incolor}{HTML}{303F9F}
    \definecolor{outcolor}{HTML}{D84315}
    \definecolor{cellborder}{HTML}{CFCFCF}
    \definecolor{cellbackground}{HTML}{F7F7F7}
    
    % prompt
    \makeatletter
    \newcommand{\boxspacing}{\kern\kvtcb@left@rule\kern\kvtcb@boxsep}
    \makeatother
    \newcommand{\prompt}[4]{
        \ttfamily\llap{{\color{#2}[#3]:\hspace{3pt}#4}}\vspace{-\baselineskip}
    }
    

    
    % Prevent overflowing lines due to hard-to-break entities
    \sloppy 
    % Setup hyperref package
    \hypersetup{
      breaklinks=true,  % so long urls are correctly broken across lines
      colorlinks=true,
      urlcolor=urlcolor,
      linkcolor=linkcolor,
      citecolor=citecolor,
      }
    % Slightly bigger margins than the latex defaults
    
    \geometry{verbose,tmargin=1in,bmargin=1in,lmargin=1in,rmargin=1in}
    
    

\begin{document}
   

\begin{titlepage}
   \begin{center}
       \vspace*{1cm}

       \textbf{\Huge Computational Complexity Assignment I}

       \vspace{1.5cm}

       \textbf{\Large Date Yao Faustin Dieudonne}\\
	   \texttt{t201d047@gunma-u.ac.jp}

       \vspace{2cm}

       \begin{tabular}{r l}
           \textbf{Teacher in Charge:} & Kazuyuki Amano\\
           \textbf{Title:}      & Professor
       \end{tabular}

       \vspace{2cm}
            
	   \vfill 

       \includegraphics[width=0.2\textwidth]{images/university}
            
       Department of Electronics and Informatics\\
       Graduate School of Science and Technology, Gunma University\\
       Japan\\
       \today
            
   \end{center}
\end{titlepage} 


    
This report is divided in two parts. In the first part, we present our
solution to the problem of formulating an instance of the kakuro puzzle
{[}1{]} as an integer programming problem. In the second part we present
a formulation of the integer partition problem also as an integer
programming problem. As we have seen during lecture, an integer program
is a linear program where all the variables have integrality
constraints. Although integer programming is know to be NP-complete, the
Gurobi Optimizer is capable of solving such models using
state-of-the-art mathematics and computer science {[}2{]}. We use the
Python API provided with the Gurobi Optimizer to formulate and solve the
problems. The jupyter notebook containing the code and the generated files for each model are available at {[}3{]}.

\section{Solving an instance of Kakuro with integer
programming}\label{solving-an-instance-of-kakuro-with-integer-programming}

The rules of the kakuro puzzle {[}1{]} are simple:

\begin{enumerate}
\def\labelenumi{\arabic{enumi}.}
\tightlist
\item
  Each cell can contain integers from 1 through 9
\item
  The clues in the black cells tell the sum of the numbers next to that
  clue. (on the right or down)
\item
  The numbers in consecutive white cells must be unique.
\end{enumerate}

The specific instance that we we solve here is given in image below.

\begin{center}
	\includegraphics[width=0.4\textwidth]{images/kakuro_instance}
\end{center}

\subsection{Model Formulation}\label{model-formulation}

\subsubsection{Decision variables}\label{decision-variables}

We choose to number the cells from top to bottom, left to right starting
from \(1\). Thus the decision variables for this model are given by
\(assign_{ijv}\) for
\((i,j,v) \in \{1,2,3,4,5,6\}^2 \times \{1,\dots,9\}\). If
\(assign_{ijv} = 1\) then the cell with coordinates \((i,j)\) is
assigned the value \(v\), otherwise \(assign_{ijv} = 0\).

\subsubsection{Constraints}\label{constraints}

The indications in the rules each give rise to some constraints.
Firstly, since each cell must be assigned exactly one number we know
that for all blank cells \((i,j)\)

\[\sum_{v=1}^{9} assign_{ijv} = 1\]

Secondly, if two cells \((i,j)\) and \((k,l)\) belong to the same
horizontal or vertical summation group then they must have different
values assigned to each of them. We express this constraint as follow:

\[assign_{ijv} + assign_{klv} \leq 1\]

for each \(v\). The inequality means that the value \(v\) may not be
assigned to any of the two cells under consideration.

Finally, there are the constraints generated from the clues in the black
cells.

\subsection{Model deployment}\label{model-deployment}

We begin by importing the Gurobi Python Module. Then we initialize some
data structures with the given data.

    \begin{tcolorbox}[breakable, size=fbox, boxrule=1pt, pad at break*=1mm,colback=cellbackground, colframe=cellborder]
\prompt{In}{incolor}{1}{\boxspacing}
\begin{Verbatim}[commandchars=\\\{\}]
\PY{k+kn}{import} \PY{n+nn}{sys}
\PY{k+kn}{import} \PY{n+nn}{itertools}

\PY{k+kn}{import} \PY{n+nn}{gurobipy} \PY{k}{as} \PY{n+nn}{gb}

\PY{c+c1}{\PYZsh{} A list of allowed values for the cells}
\PY{n}{values} \PY{o}{=} \PY{n+nb}{list}\PY{p}{(}\PY{n+nb}{range}\PY{p}{(}\PY{l+m+mi}{1}\PY{p}{,}\PY{l+m+mi}{10}\PY{p}{)}\PY{p}{)}

\PY{c+c1}{\PYZsh{} A list of the blank cells}
\PY{n}{cells} \PY{o}{=} \PY{p}{[}
           \PY{p}{(}\PY{l+m+mi}{2}\PY{p}{,}\PY{l+m+mi}{3}\PY{p}{)}\PY{p}{,} \PY{p}{(}\PY{l+m+mi}{2}\PY{p}{,}\PY{l+m+mi}{4}\PY{p}{)}\PY{p}{,}
    \PY{p}{(}\PY{l+m+mi}{3}\PY{p}{,}\PY{l+m+mi}{2}\PY{p}{)}\PY{p}{,} \PY{p}{(}\PY{l+m+mi}{3}\PY{p}{,}\PY{l+m+mi}{3}\PY{p}{)}\PY{p}{,} \PY{p}{(}\PY{l+m+mi}{3}\PY{p}{,}\PY{l+m+mi}{4}\PY{p}{)}\PY{p}{,} \PY{p}{(}\PY{l+m+mi}{3}\PY{p}{,}\PY{l+m+mi}{5}\PY{p}{)}\PY{p}{,} \PY{p}{(}\PY{l+m+mi}{3}\PY{p}{,}\PY{l+m+mi}{6}\PY{p}{)}\PY{p}{,}
    \PY{p}{(}\PY{l+m+mi}{4}\PY{p}{,}\PY{l+m+mi}{2}\PY{p}{)}\PY{p}{,} \PY{p}{(}\PY{l+m+mi}{4}\PY{p}{,}\PY{l+m+mi}{3}\PY{p}{)}\PY{p}{,}        \PY{p}{(}\PY{l+m+mi}{4}\PY{p}{,}\PY{l+m+mi}{5}\PY{p}{)}\PY{p}{,} \PY{p}{(}\PY{l+m+mi}{4}\PY{p}{,}\PY{l+m+mi}{6}\PY{p}{)}\PY{p}{,}
    \PY{p}{(}\PY{l+m+mi}{5}\PY{p}{,}\PY{l+m+mi}{2}\PY{p}{)}\PY{p}{,} \PY{p}{(}\PY{l+m+mi}{5}\PY{p}{,}\PY{l+m+mi}{3}\PY{p}{)}\PY{p}{,} \PY{p}{(}\PY{l+m+mi}{5}\PY{p}{,}\PY{l+m+mi}{4}\PY{p}{)}\PY{p}{,} \PY{p}{(}\PY{l+m+mi}{5}\PY{p}{,}\PY{l+m+mi}{5}\PY{p}{)}\PY{p}{,} \PY{p}{(}\PY{l+m+mi}{5}\PY{p}{,}\PY{l+m+mi}{6}\PY{p}{)}\PY{p}{,}
                  \PY{p}{(}\PY{l+m+mi}{6}\PY{p}{,}\PY{l+m+mi}{4}\PY{p}{)}\PY{p}{,} \PY{p}{(}\PY{l+m+mi}{6}\PY{p}{,}\PY{l+m+mi}{5}\PY{p}{)}
\PY{p}{]}

\PY{c+c1}{\PYZsh{} An encoding of the horizontal clues.}
\PY{c+c1}{\PYZsh{} The values for the cells in each sublist must must add up to the key.}
\PY{n}{data\PYZus{}horizontal} \PY{o}{=} \PY{p}{\PYZob{}}
    \PY{l+m+mi}{3}\PY{p}{:} \PY{p}{[}\PY{p}{[}\PY{p}{(}\PY{l+m+mi}{2}\PY{p}{,}\PY{l+m+mi}{3}\PY{p}{)}\PY{p}{,} \PY{p}{(}\PY{l+m+mi}{2}\PY{p}{,}\PY{l+m+mi}{4}\PY{p}{)}\PY{p}{]}\PY{p}{,} 
        \PY{p}{[}\PY{p}{(}\PY{l+m+mi}{4}\PY{p}{,}\PY{l+m+mi}{2}\PY{p}{)}\PY{p}{,} \PY{p}{(}\PY{l+m+mi}{4}\PY{p}{,}\PY{l+m+mi}{3}\PY{p}{)}\PY{p}{]}\PY{p}{]}\PY{p}{,}
    \PY{l+m+mi}{16}\PY{p}{:} \PY{p}{[}\PY{p}{[}\PY{p}{(}\PY{l+m+mi}{3}\PY{p}{,}\PY{l+m+mi}{2}\PY{p}{)}\PY{p}{,} \PY{p}{(}\PY{l+m+mi}{3}\PY{p}{,}\PY{l+m+mi}{3}\PY{p}{)}\PY{p}{,} \PY{p}{(}\PY{l+m+mi}{3}\PY{p}{,}\PY{l+m+mi}{4}\PY{p}{)}\PY{p}{,} \PY{p}{(}\PY{l+m+mi}{3}\PY{p}{,}\PY{l+m+mi}{5}\PY{p}{)}\PY{p}{,} \PY{p}{(}\PY{l+m+mi}{3}\PY{p}{,}\PY{l+m+mi}{6}\PY{p}{)}\PY{p}{]}\PY{p}{,}
         \PY{p}{[}\PY{p}{(}\PY{l+m+mi}{6}\PY{p}{,}\PY{l+m+mi}{4}\PY{p}{)}\PY{p}{,} \PY{p}{(}\PY{l+m+mi}{6}\PY{p}{,}\PY{l+m+mi}{5}\PY{p}{)}\PY{p}{]}\PY{p}{]}\PY{p}{,}
    \PY{l+m+mi}{13}\PY{p}{:} \PY{p}{[}\PY{p}{[}\PY{p}{(}\PY{l+m+mi}{4}\PY{p}{,}\PY{l+m+mi}{5}\PY{p}{)}\PY{p}{,} \PY{p}{(}\PY{l+m+mi}{4}\PY{p}{,}\PY{l+m+mi}{6}\PY{p}{)}\PY{p}{]}\PY{p}{]}\PY{p}{,}
    \PY{l+m+mi}{17}\PY{p}{:} \PY{p}{[}\PY{p}{[}\PY{p}{(}\PY{l+m+mi}{5}\PY{p}{,}\PY{l+m+mi}{2}\PY{p}{)}\PY{p}{,} \PY{p}{(}\PY{l+m+mi}{5}\PY{p}{,}\PY{l+m+mi}{3}\PY{p}{)}\PY{p}{,} \PY{p}{(}\PY{l+m+mi}{5}\PY{p}{,}\PY{l+m+mi}{4}\PY{p}{)}\PY{p}{,} \PY{p}{(}\PY{l+m+mi}{5}\PY{p}{,}\PY{l+m+mi}{5}\PY{p}{)}\PY{p}{,} \PY{p}{(}\PY{l+m+mi}{5}\PY{p}{,}\PY{l+m+mi}{6}\PY{p}{)}\PY{p}{]}\PY{p}{]}
\PY{p}{\PYZcb{}}

\PY{c+c1}{\PYZsh{} An encoding of the vertical clues.}
\PY{c+c1}{\PYZsh{} The values for the cells in each sublist must must add up to the key.}
\PY{n}{data\PYZus{}vertical} \PY{o}{=} \PY{p}{\PYZob{}}
    \PY{l+m+mi}{7}\PY{p}{:} \PY{p}{[}\PY{p}{[}\PY{p}{(}\PY{l+m+mi}{3}\PY{p}{,}\PY{l+m+mi}{2}\PY{p}{)}\PY{p}{,} \PY{p}{(}\PY{l+m+mi}{4}\PY{p}{,}\PY{l+m+mi}{2}\PY{p}{)}\PY{p}{,} \PY{p}{(}\PY{l+m+mi}{5}\PY{p}{,}\PY{l+m+mi}{2}\PY{p}{)}\PY{p}{]}\PY{p}{]}\PY{p}{,}
    \PY{l+m+mi}{10}\PY{p}{:} \PY{p}{[}\PY{p}{[}\PY{p}{(}\PY{l+m+mi}{2}\PY{p}{,}\PY{l+m+mi}{3}\PY{p}{)}\PY{p}{,} \PY{p}{(}\PY{l+m+mi}{3}\PY{p}{,}\PY{l+m+mi}{3}\PY{p}{)}\PY{p}{,} \PY{p}{(}\PY{l+m+mi}{4}\PY{p}{,}\PY{l+m+mi}{3}\PY{p}{)}\PY{p}{,} \PY{p}{(}\PY{l+m+mi}{5}\PY{p}{,}\PY{l+m+mi}{3}\PY{p}{)}\PY{p}{]}\PY{p}{]}\PY{p}{,}
    \PY{l+m+mi}{4}\PY{p}{:} \PY{p}{[}\PY{p}{[}\PY{p}{(}\PY{l+m+mi}{2}\PY{p}{,}\PY{l+m+mi}{4}\PY{p}{)}\PY{p}{,} \PY{p}{(}\PY{l+m+mi}{3}\PY{p}{,}\PY{l+m+mi}{4}\PY{p}{)}\PY{p}{]}\PY{p}{]}\PY{p}{,}
    \PY{l+m+mi}{9}\PY{p}{:} \PY{p}{[}\PY{p}{[}\PY{p}{(}\PY{l+m+mi}{5}\PY{p}{,}\PY{l+m+mi}{4}\PY{p}{)}\PY{p}{,} \PY{p}{(}\PY{l+m+mi}{6}\PY{p}{,}\PY{l+m+mi}{4}\PY{p}{)}\PY{p}{]}\PY{p}{]}\PY{p}{,}
    \PY{l+m+mi}{30}\PY{p}{:} \PY{p}{[}\PY{p}{[}\PY{p}{(}\PY{l+m+mi}{3}\PY{p}{,}\PY{l+m+mi}{5}\PY{p}{)}\PY{p}{,} \PY{p}{(}\PY{l+m+mi}{4}\PY{p}{,}\PY{l+m+mi}{5}\PY{p}{)}\PY{p}{,} \PY{p}{(}\PY{l+m+mi}{5}\PY{p}{,}\PY{l+m+mi}{5}\PY{p}{)}\PY{p}{,} \PY{p}{(}\PY{l+m+mi}{6}\PY{p}{,}\PY{l+m+mi}{5}\PY{p}{)}\PY{p}{]}\PY{p}{]}\PY{p}{,}
    \PY{l+m+mi}{8}\PY{p}{:} \PY{p}{[}\PY{p}{[}\PY{p}{(}\PY{l+m+mi}{3}\PY{p}{,}\PY{l+m+mi}{6}\PY{p}{)}\PY{p}{,} \PY{p}{(}\PY{l+m+mi}{4}\PY{p}{,}\PY{l+m+mi}{6}\PY{p}{)}\PY{p}{,} \PY{p}{(}\PY{l+m+mi}{5}\PY{p}{,}\PY{l+m+mi}{6}\PY{p}{)}\PY{p}{]}\PY{p}{]}
\PY{p}{\PYZcb{}}
\end{Verbatim}
\end{tcolorbox}

    Next, we instanciate a model, declare the decision variables and add the
necessary constraints. We set the objective function to a constant. Then
we instruct the Gurobi Optimizer to find an assignment of the variables
that sasisfies the constraints.

    \begin{tcolorbox}[breakable, size=fbox, boxrule=1pt, pad at break*=1mm,colback=cellbackground, colframe=cellborder]
\prompt{In}{incolor}{2}{\boxspacing}
\begin{Verbatim}[commandchars=\\\{\}]
\PY{c+c1}{\PYZsh{} declare and initialize model}
\PY{n}{m} \PY{o}{=} \PY{n}{gb}\PY{o}{.}\PY{n}{Model}\PY{p}{(}\PY{l+s+s1}{\PYZsq{}}\PY{l+s+s1}{kakuro}\PY{l+s+s1}{\PYZsq{}}\PY{p}{)}

\PY{c+c1}{\PYZsh{} Decision variables for cell values}
\PY{n}{assign} \PY{o}{=} \PY{n}{m}\PY{o}{.}\PY{n}{addVars}\PY{p}{(}\PY{n}{cells}\PY{p}{,} \PY{n}{values}\PY{p}{,} \PY{n}{name}\PY{o}{=}\PY{l+s+s2}{\PYZdq{}}\PY{l+s+s2}{assign}\PY{l+s+s2}{\PYZdq{}}\PY{p}{,} \PY{n}{vtype}\PY{o}{=}\PY{n}{gb}\PY{o}{.}\PY{n}{GRB}\PY{o}{.}\PY{n}{BINARY}\PY{p}{)}

\PY{c+c1}{\PYZsh{} Each cell must be assigned exactly one value}
\PY{n}{unique\PYZus{}assignement} \PY{o}{=} \PY{n}{m}\PY{o}{.}\PY{n}{addConstrs}\PY{p}{(}\PY{p}{(}\PY{n}{assign}\PY{o}{.}\PY{n}{sum}\PY{p}{(}\PY{n}{x}\PY{p}{,} \PY{n}{y}\PY{p}{,} \PY{l+s+s1}{\PYZsq{}}\PY{l+s+s1}{*}\PY{l+s+s1}{\PYZsq{}}\PY{p}{)} \PY{o}{==} \PY{l+m+mi}{1}
                                  \PY{k}{for} \PY{n}{x}\PY{p}{,} \PY{n}{y} \PY{o+ow}{in} \PY{n}{cells}\PY{p}{)}\PY{p}{,} \PY{l+s+s1}{\PYZsq{}}\PY{l+s+s1}{unique\PYZus{}assignment}\PY{l+s+s1}{\PYZsq{}}\PY{p}{)}

\PY{c+c1}{\PYZsh{} Horizontal clues constraints}
\PY{k}{for} \PY{n}{total}\PY{p}{,} \PY{n}{lists} \PY{o+ow}{in} \PY{n}{data\PYZus{}horizontal}\PY{o}{.}\PY{n}{items}\PY{p}{(}\PY{p}{)}\PY{p}{:}
    \PY{k}{for} \PY{n}{i}\PY{p}{,} \PY{n}{l} \PY{o+ow}{in} \PY{n+nb}{enumerate}\PY{p}{(}\PY{n}{lists}\PY{p}{)}\PY{p}{:}
        \PY{n}{expr} \PY{o}{=} \PY{n}{gb}\PY{o}{.}\PY{n}{LinExpr}\PY{p}{(}\PY{p}{)}
        \PY{k}{for} \PY{n}{row}\PY{p}{,} \PY{n}{col} \PY{o+ow}{in} \PY{n}{l}\PY{p}{:}
            \PY{k}{for} \PY{n}{v} \PY{o+ow}{in} \PY{n}{values}\PY{p}{:}
                \PY{n}{expr}\PY{o}{.}\PY{n}{add}\PY{p}{(}\PY{n}{assign}\PY{p}{[}\PY{n}{row}\PY{p}{,} \PY{n}{col}\PY{p}{,} \PY{n}{v}\PY{p}{]}\PY{o}{*}\PY{n}{v}\PY{p}{)}
        \PY{n}{m}\PY{o}{.}\PY{n}{addConstr}\PY{p}{(}\PY{n}{expr} \PY{o}{==} \PY{n}{total}\PY{p}{,} \PY{l+s+s2}{\PYZdq{}}\PY{l+s+s2}{hor\PYZus{}}\PY{l+s+si}{\PYZob{}\PYZcb{}}\PY{l+s+s2}{\PYZus{}}\PY{l+s+si}{\PYZob{}\PYZcb{}}\PY{l+s+s2}{\PYZdq{}}\PY{o}{.}\PY{n}{format}\PY{p}{(}\PY{n}{total}\PY{p}{,} \PY{n}{i}\PY{p}{)}\PY{p}{)}
        
\PY{c+c1}{\PYZsh{} Vertical clues constraints}
\PY{k}{for} \PY{n}{total}\PY{p}{,} \PY{n}{lists} \PY{o+ow}{in} \PY{n}{data\PYZus{}vertical}\PY{o}{.}\PY{n}{items}\PY{p}{(}\PY{p}{)}\PY{p}{:}
    \PY{k}{for} \PY{n}{i}\PY{p}{,} \PY{n}{l} \PY{o+ow}{in} \PY{n+nb}{enumerate}\PY{p}{(}\PY{n}{lists}\PY{p}{)}\PY{p}{:}
        \PY{n}{expr} \PY{o}{=} \PY{n}{gb}\PY{o}{.}\PY{n}{LinExpr}\PY{p}{(}\PY{p}{)}
        \PY{k}{for} \PY{n}{row}\PY{p}{,} \PY{n}{col} \PY{o+ow}{in} \PY{n}{l}\PY{p}{:}
            \PY{k}{for} \PY{n}{v} \PY{o+ow}{in} \PY{n}{values}\PY{p}{:}
                \PY{n}{expr}\PY{o}{.}\PY{n}{add}\PY{p}{(}\PY{n}{assign}\PY{p}{[}\PY{n}{row}\PY{p}{,} \PY{n}{col}\PY{p}{,} \PY{n}{v}\PY{p}{]}\PY{o}{*}\PY{n}{v}\PY{p}{)}
        \PY{n}{m}\PY{o}{.}\PY{n}{addConstr}\PY{p}{(}\PY{n}{expr} \PY{o}{==} \PY{n}{total}\PY{p}{,} \PY{l+s+s2}{\PYZdq{}}\PY{l+s+s2}{vert\PYZus{}}\PY{l+s+si}{\PYZob{}\PYZcb{}}\PY{l+s+s2}{\PYZus{}}\PY{l+s+si}{\PYZob{}\PYZcb{}}\PY{l+s+s2}{\PYZdq{}}\PY{o}{.}\PY{n}{format}\PY{p}{(}\PY{n}{total}\PY{p}{,} \PY{n}{i}\PY{p}{)}\PY{p}{)}
        
\PY{c+c1}{\PYZsh{} Constraints for unicity of values in consecutive white cells}
\PY{k}{for} \PY{n}{lists} \PY{o+ow}{in} \PY{n}{data\PYZus{}vertical}\PY{o}{.}\PY{n}{values}\PY{p}{(}\PY{p}{)}\PY{p}{:}
    \PY{k}{for} \PY{n}{l} \PY{o+ow}{in} \PY{n}{lists}\PY{p}{:}
        \PY{k}{for} \PY{p}{(}\PY{n}{x1}\PY{p}{,} \PY{n}{y1}\PY{p}{)}\PY{p}{,} \PY{p}{(}\PY{n}{x2}\PY{p}{,} \PY{n}{y2}\PY{p}{)} \PY{o+ow}{in} \PY{n}{itertools}\PY{o}{.}\PY{n}{combinations}\PY{p}{(}\PY{n}{l}\PY{p}{,} \PY{l+m+mi}{2}\PY{p}{)}\PY{p}{:}
            \PY{n}{m}\PY{o}{.}\PY{n}{addConstrs}\PY{p}{(}\PY{n}{assign}\PY{p}{[}\PY{n}{x1}\PY{p}{,} \PY{n}{y1}\PY{p}{,} \PY{n}{v}\PY{p}{]} \PY{o}{+} \PY{n}{assign}\PY{p}{[}\PY{n}{x2}\PY{p}{,} \PY{n}{y2}\PY{p}{,} \PY{n}{v}\PY{p}{]} \PY{o}{\PYZlt{}}\PY{o}{=} \PY{l+m+mi}{1} 
                        \PY{k}{for} \PY{n}{v} \PY{o+ow}{in} \PY{n}{values}\PY{p}{)}
            
\PY{k}{for} \PY{n}{lists} \PY{o+ow}{in} \PY{n}{data\PYZus{}horizontal}\PY{o}{.}\PY{n}{values}\PY{p}{(}\PY{p}{)}\PY{p}{:}
    \PY{k}{for} \PY{n}{l} \PY{o+ow}{in} \PY{n}{lists}\PY{p}{:}
        \PY{k}{for} \PY{p}{(}\PY{n}{x1}\PY{p}{,} \PY{n}{y1}\PY{p}{)}\PY{p}{,} \PY{p}{(}\PY{n}{x2}\PY{p}{,} \PY{n}{y2}\PY{p}{)} \PY{o+ow}{in} \PY{n}{itertools}\PY{o}{.}\PY{n}{combinations}\PY{p}{(}\PY{n}{l}\PY{p}{,} \PY{l+m+mi}{2}\PY{p}{)}\PY{p}{:}
            \PY{n}{m}\PY{o}{.}\PY{n}{addConstrs}\PY{p}{(}\PY{n}{assign}\PY{p}{[}\PY{n}{x1}\PY{p}{,} \PY{n}{y1}\PY{p}{,} \PY{n}{v}\PY{p}{]} \PY{o}{+} \PY{n}{assign}\PY{p}{[}\PY{n}{x2}\PY{p}{,} \PY{n}{y2}\PY{p}{,} \PY{n}{v}\PY{p}{]} \PY{o}{\PYZlt{}}\PY{o}{=} \PY{l+m+mi}{1} 
                        \PY{k}{for} \PY{n}{v} \PY{o+ow}{in} \PY{n}{values}\PY{p}{)}
            
\PY{c+c1}{\PYZsh{} Set objective function to a constant}
\PY{n}{m}\PY{o}{.}\PY{n}{setObjective}\PY{p}{(}\PY{l+m+mi}{1}\PY{p}{)}

\PY{c+c1}{\PYZsh{} Save generated model for inspection.}
\PY{c+c1}{\PYZsh{} available at https://github.com/faustind/gurobipy/blob/master/src/files/kakuro.lp}
\PY{n}{m}\PY{o}{.}\PY{n}{write}\PY{p}{(}\PY{l+s+s1}{\PYZsq{}}\PY{l+s+s1}{files/kakuro.lp}\PY{l+s+s1}{\PYZsq{}}\PY{p}{)}

\PY{n}{m}\PY{o}{.}\PY{n}{optimize}\PY{p}{(}\PY{p}{)}
\end{Verbatim}
\end{tcolorbox}

    \begin{Verbatim}[commandchars=\\\{\}]
Using license file /home/faustind/gurobi.lic
Academic license - for non-commercial use only
Gurobi Optimizer version 9.0.2 build v9.0.2rc0 (linux64)
Optimize a model with 426 rows, 162 columns and 1278 nonzeros
Model fingerprint: 0x5640372b
Variable types: 0 continuous, 162 integer (162 binary)
Coefficient statistics:
  Matrix range     [1e+00, 9e+00]
  Objective range  [0e+00, 0e+00]
  Bounds range     [1e+00, 1e+00]
  RHS range        [1e+00, 3e+01]
Presolve removed 426 rows and 162 columns
Presolve time: 0.00s
Presolve: All rows and columns removed

Explored 0 nodes (0 simplex iterations) in 0.03 seconds
Thread count was 1 (of 4 available processors)

Solution count 1: 1

Optimal solution found (tolerance 1.00e-04)
Best objective 1.000000000000e+00, best bound 1.000000000000e+00, gap 0.0000\%
    \end{Verbatim}

    \subsection{Cell Assignment}\label{cell-assignment}

The optimization results in the assignment of appropriate values to each
cell.

    \begin{tcolorbox}[breakable, size=fbox, boxrule=1pt, pad at break*=1mm,colback=cellbackground, colframe=cellborder]
\prompt{In}{incolor}{3}{\boxspacing}
\begin{Verbatim}[commandchars=\\\{\}]
\PY{c+c1}{\PYZsh{} display the values assigned to each cell}
\PY{k}{for} \PY{n}{i}\PY{p}{,}\PY{n}{j}\PY{p}{,}\PY{n}{v} \PY{o+ow}{in} \PY{n}{assign}\PY{o}{.}\PY{n}{keys}\PY{p}{(}\PY{p}{)}\PY{p}{:}
    \PY{k}{if} \PY{p}{(}\PY{n+nb}{abs}\PY{p}{(}\PY{n}{assign}\PY{p}{[}\PY{n}{i}\PY{p}{,}\PY{n}{j}\PY{p}{,}\PY{n}{v}\PY{p}{]}\PY{o}{.}\PY{n}{x}\PY{p}{)} \PY{o}{\PYZgt{}}\PY{o}{=} \PY{n}{sys}\PY{o}{.}\PY{n}{float\PYZus{}info}\PY{o}{.}\PY{n}{epsilon}\PY{p}{)}\PY{p}{:}
        \PY{n+nb}{print}\PY{p}{(}\PY{l+s+s1}{\PYZsq{}}\PY{l+s+s1}{Cell(}\PY{l+s+si}{\PYZob{}\PYZcb{}}\PY{l+s+s1}{, }\PY{l+s+si}{\PYZob{}\PYZcb{}}\PY{l+s+s1}{) = }\PY{l+s+si}{\PYZob{}\PYZcb{}}\PY{l+s+s1}{\PYZsq{}}\PY{o}{.}\PY{n}{format}\PY{p}{(}\PY{n}{i}\PY{p}{,}\PY{n}{j}\PY{p}{,}\PY{n}{v}\PY{p}{)}\PY{p}{)}
\end{Verbatim}
\end{tcolorbox}

    \begin{Verbatim}[commandchars=\\\{\}]
Cell(2, 3) = 2
Cell(2, 4) = 1
Cell(3, 2) = 1
Cell(3, 3) = 4
Cell(3, 4) = 3
Cell(3, 5) = 6
Cell(3, 6) = 2
Cell(4, 2) = 2
Cell(4, 3) = 1
Cell(4, 5) = 8
Cell(4, 6) = 5
Cell(5, 2) = 4
Cell(5, 3) = 3
Cell(5, 4) = 2
Cell(5, 5) = 7
Cell(5, 6) = 1
Cell(6, 4) = 7
Cell(6, 5) = 9
    \end{Verbatim}

    \subsection{Puzzle Solution}\label{puzzle-solution}

From the above assignments, we can solve the puzzle as shown in the
image below.

\begin{center}
	\includegraphics[width=0.4\textwidth]{images/kakuro_instance_assignment}
\end{center}

    \section{The Integer Partition
Problem}\label{the-integer-partition-problem}

In the \emph{integer partition} problem we seek to partition the
elements of a set \(S\) in two sets \(A\) and \(B\) such that
\(\sum_{a \in A} a = \sum_{b \in B} b\) or alternatively make the
difference as small as possible {[}4{]}.

\subsection{Model Formulation}\label{model-formulation}

\subsubsection{Decision Variables}\label{decision-variables}

For each \(x_i \in S\) we create two variables \(in_{A}^i\) and
\(in_{B}^i\). These variables can only take values in \(\{0,1\}\). And
\(in_{P}^i = 1\) if and only if \(x_i \in P\) for \(P \in \{A, B\}\).

\subsubsection{Constraints}\label{constraints}

An element can only belong to one set in the resulting partition. So for
each \(x_i \in S\)

\[in_A^i + in_B^i = 1.\]

In addition, the difference must be minimized, but remain greater than
0.

\[\sum_{a \in A}a - \sum_{b \in B}b \geq 0\]

\subsubsection{Objective}\label{objective}

Since an exact solution to the partition problem does not always
exists, we will try to minimize the difference of the two partitions.
When there is an exact solution, the difference is equal to \(0\) and it is
the smallest it can get because of the non-negativity constraint.

\[Minimize \sum_{a \in A}a - \sum_{b \in B}b\]

\subsection{Model Deployment}\label{model-deployment}

We begin by importing the necessary modules. Next we define a reusable
function to run a model. The function \texttt{integer\_partition} takes
a list of integers defining an instance of the integer partition
problem. It declares the decision variables then adds the
constraints. The function \texttt{print\_solution} just prints a
solution returned by \texttt{integer\_partition}.

    \begin{tcolorbox}[breakable, size=fbox, boxrule=1pt, pad at break*=1mm,colback=cellbackground, colframe=cellborder]
\prompt{In}{incolor}{4}{\boxspacing}
\begin{Verbatim}[commandchars=\\\{\}]
\PY{k+kn}{import} \PY{n+nn}{sys}
\PY{k+kn}{import} \PY{n+nn}{random}
\PY{k+kn}{import} \PY{n+nn}{gurobipy} \PY{k}{as} \PY{n+nn}{gb}

\PY{k+kn}{from} \PY{n+nn}{collections} \PY{k+kn}{import} \PY{n}{defaultdict}
\PY{k+kn}{from} \PY{n+nn}{gurobipy} \PY{k+kn}{import} \PY{n}{GRB}


\PY{k}{def} \PY{n+nf}{integer\PYZus{}partition}\PY{p}{(}\PY{n}{S}\PY{p}{,} \PY{n}{fname}\PY{o}{=}\PY{k+kc}{None}\PY{p}{)}\PY{p}{:}
    \PY{l+s+sd}{\PYZdq{}\PYZdq{}\PYZdq{}Return a dictionary solution to the integer partition on S.}
\PY{l+s+sd}{    }
\PY{l+s+sd}{    Args:}
\PY{l+s+sd}{        S: iterable of integers}
\PY{l+s+sd}{        fname: OPTIONAL output file for the generated model}
\PY{l+s+sd}{    \PYZdq{}\PYZdq{}\PYZdq{}}
    
    \PY{n}{N} \PY{o}{=} \PY{n+nb}{len}\PY{p}{(}\PY{n}{S}\PY{p}{)}
    
    \PY{n}{m} \PY{o}{=} \PY{n}{gb}\PY{o}{.}\PY{n}{Model}\PY{p}{(}\PY{l+s+s1}{\PYZsq{}}\PY{l+s+s1}{IP\PYZus{}}\PY{l+s+si}{\PYZob{}\PYZcb{}}\PY{l+s+s1}{\PYZsq{}}\PY{o}{.}\PY{n}{format}\PY{p}{(}\PY{n}{fname}\PY{p}{)}\PY{p}{)}

    \PY{c+c1}{\PYZsh{} decision variables for model integer partition}
    \PY{n}{partition} \PY{o}{=} \PY{n}{m}\PY{o}{.}\PY{n}{addVars}\PY{p}{(}\PY{p}{[}\PY{l+s+s1}{\PYZsq{}}\PY{l+s+s1}{A}\PY{l+s+s1}{\PYZsq{}}\PY{p}{,} \PY{l+s+s1}{\PYZsq{}}\PY{l+s+s1}{B}\PY{l+s+s1}{\PYZsq{}}\PY{p}{]}\PY{p}{,} \PY{n+nb}{range}\PY{p}{(}\PY{n}{N}\PY{p}{)}
                          \PY{p}{,} \PY{n}{name}\PY{o}{=}\PY{l+s+s2}{\PYZdq{}}\PY{l+s+s2}{in}\PY{l+s+s2}{\PYZdq{}}
                          \PY{p}{,} \PY{n}{vtype}\PY{o}{=}\PY{n}{GRB}\PY{o}{.}\PY{n}{BINARY}\PY{p}{)}

    \PY{c+c1}{\PYZsh{} each elements must be in exactly one set in the partition}
    \PY{n}{separation} \PY{o}{=} \PY{n}{m}\PY{o}{.}\PY{n}{addConstrs}\PY{p}{(}\PY{p}{(}\PY{n}{partition}\PY{p}{[}\PY{l+s+s1}{\PYZsq{}}\PY{l+s+s1}{A}\PY{l+s+s1}{\PYZsq{}}\PY{p}{,} \PY{n}{i}\PY{p}{]} \PY{o}{+} \PY{n}{partition}\PY{p}{[}\PY{l+s+s1}{\PYZsq{}}\PY{l+s+s1}{B}\PY{l+s+s1}{\PYZsq{}}\PY{p}{,} \PY{n}{i}\PY{p}{]} \PY{o}{==} \PY{l+m+mi}{1} \PY{k}{for} \PY{n}{i} \PY{o+ow}{in} \PY{n+nb}{range}\PY{p}{(}\PY{n}{N}\PY{p}{)}\PY{p}{)}
                              \PY{p}{,} \PY{n}{name}\PY{o}{=}\PY{l+s+s2}{\PYZdq{}}\PY{l+s+s2}{separation}\PY{l+s+s2}{\PYZdq{}}\PY{p}{)}

    \PY{c+c1}{\PYZsh{} Constraint to keep the difference non negative}
    \PY{n}{min\PYZus{}diff} \PY{o}{=} \PY{n}{m}\PY{o}{.}\PY{n}{addConstr}\PY{p}{(}\PY{n+nb}{sum}\PY{p}{(}\PY{n}{S}\PY{p}{[}\PY{n}{i}\PY{p}{]}\PY{o}{*}\PY{n}{partition}\PY{p}{[}\PY{l+s+s1}{\PYZsq{}}\PY{l+s+s1}{A}\PY{l+s+s1}{\PYZsq{}}\PY{p}{,}\PY{n}{i}\PY{p}{]} 
                                \PY{k}{for} \PY{n}{i} \PY{o+ow}{in} \PY{n+nb}{range}\PY{p}{(}\PY{n}{N}\PY{p}{)}\PY{p}{)}
                            \PY{o}{\PYZhy{}}\PY{n+nb}{sum}\PY{p}{(}\PY{n}{S}\PY{p}{[}\PY{n}{i}\PY{p}{]}\PY{o}{*}\PY{n}{partition}\PY{p}{[}\PY{l+s+s1}{\PYZsq{}}\PY{l+s+s1}{B}\PY{l+s+s1}{\PYZsq{}}\PY{p}{,}\PY{n}{i}\PY{p}{]} 
                                 \PY{k}{for} \PY{n}{i} \PY{o+ow}{in} \PY{n+nb}{range}\PY{p}{(}\PY{n}{N}\PY{p}{)}\PY{p}{)} \PY{o}{\PYZgt{}}\PY{o}{=} \PY{l+m+mi}{0}
                          \PY{p}{,} \PY{n}{name}\PY{o}{=}\PY{l+s+s1}{\PYZsq{}}\PY{l+s+s1}{minimize\PYZus{}difference}\PY{l+s+s1}{\PYZsq{}}\PY{p}{)}

    \PY{n}{m}\PY{o}{.}\PY{n}{setObjective}\PY{p}{(}\PY{n+nb}{sum}\PY{p}{(}\PY{n}{S}\PY{p}{[}\PY{n}{i}\PY{p}{]}\PY{o}{*}\PY{n}{partition}\PY{p}{[}\PY{l+s+s1}{\PYZsq{}}\PY{l+s+s1}{A}\PY{l+s+s1}{\PYZsq{}}\PY{p}{,}\PY{n}{i}\PY{p}{]} \PY{k}{for} \PY{n}{i} \PY{o+ow}{in} \PY{n+nb}{range}\PY{p}{(}\PY{n}{N}\PY{p}{)}\PY{p}{)}
                   \PY{o}{\PYZhy{}} \PY{n+nb}{sum}\PY{p}{(}\PY{n}{S}\PY{p}{[}\PY{n}{i}\PY{p}{]}\PY{o}{*}\PY{n}{partition}\PY{p}{[}\PY{l+s+s1}{\PYZsq{}}\PY{l+s+s1}{B}\PY{l+s+s1}{\PYZsq{}}\PY{p}{,}\PY{n}{i}\PY{p}{]} \PY{k}{for} \PY{n}{i} \PY{o+ow}{in} \PY{n+nb}{range}\PY{p}{(}\PY{n}{N}\PY{p}{)}\PY{p}{)}
                   \PY{p}{,} \PY{n}{GRB}\PY{o}{.}\PY{n}{MINIMIZE}\PY{p}{)}

    \PY{c+c1}{\PYZsh{} Save generated model for inspection.}
    \PY{n}{m}\PY{o}{.}\PY{n}{write}\PY{p}{(}\PY{l+s+s1}{\PYZsq{}}\PY{l+s+s1}{files/}\PY{l+s+si}{\PYZob{}\PYZcb{}}\PY{l+s+s1}{\PYZsq{}}\PY{o}{.}\PY{n}{format}\PY{p}{(}\PY{n}{fname} \PY{o+ow}{or} \PY{l+s+s1}{\PYZsq{}}\PY{l+s+s1}{integer\PYZus{}partition.lp}\PY{l+s+s1}{\PYZsq{}}\PY{p}{)}\PY{p}{)}
    \PY{n}{m}\PY{o}{.}\PY{n}{optimize}\PY{p}{(}\PY{p}{)}
    \PY{n}{result} \PY{o}{=} \PY{n}{defaultdict}\PY{p}{(}\PY{n+nb}{list}\PY{p}{)}

    \PY{c+c1}{\PYZsh{} add each element of S to the set to which it belongs in the partition}
    \PY{k}{for} \PY{n}{s}\PY{p}{,} \PY{n}{i} \PY{o+ow}{in} \PY{n}{partition}\PY{p}{:}
        \PY{k}{if} \PY{n}{partition}\PY{p}{[}\PY{n}{s}\PY{p}{,} \PY{n}{i}\PY{p}{]}\PY{o}{.}\PY{n}{x} \PY{o}{\PYZgt{}}\PY{o}{=} \PY{n}{sys}\PY{o}{.}\PY{n}{float\PYZus{}info}\PY{o}{.}\PY{n}{epsilon}\PY{p}{:}
            \PY{n}{result}\PY{p}{[}\PY{n}{s}\PY{p}{]}\PY{o}{.}\PY{n}{append}\PY{p}{(}\PY{n}{S}\PY{p}{[}\PY{n}{i}\PY{p}{]}\PY{p}{)}
            
    \PY{k}{return} \PY{n}{result}


\PY{k}{def} \PY{n+nf}{print\PYZus{}solution}\PY{p}{(}\PY{n}{result}\PY{p}{)}\PY{p}{:}
    \PY{l+s+sd}{\PYZdq{}\PYZdq{}\PYZdq{}Print a solution to the integer partition problem.\PYZdq{}\PYZdq{}\PYZdq{}}
    \PY{k}{for} \PY{n}{s}\PY{p}{,} \PY{n}{items} \PY{o+ow}{in} \PY{n}{result}\PY{o}{.}\PY{n}{items}\PY{p}{(}\PY{p}{)}\PY{p}{:}
        \PY{n+nb}{print}\PY{p}{(}\PY{l+s+s1}{\PYZsq{}}\PY{l+s+s1}{sum(}\PY{l+s+si}{\PYZob{}\PYZcb{}}\PY{l+s+s1}{) = sum(}\PY{l+s+si}{\PYZob{}\PYZcb{}}\PY{l+s+s1}{) = }\PY{l+s+si}{\PYZob{}\PYZcb{}}\PY{l+s+s1}{\PYZsq{}}\PY{o}{.}\PY{n}{format}\PY{p}{(}\PY{n}{s}\PY{p}{,} \PY{n}{items}\PY{p}{,} \PY{n+nb}{sum}\PY{p}{(}\PY{n}{items}\PY{p}{)}\PY{p}{)}\PY{p}{)}
\end{Verbatim}
\end{tcolorbox}

    \subsection{Running the model}\label{running-the-model}

We now run the model on two instances, one for which there is 
an exact solution and another instance with no exact
solution.

    \begin{tcolorbox}[breakable, size=fbox, boxrule=1pt, pad at break*=1mm,colback=cellbackground, colframe=cellborder]
\prompt{In}{incolor}{5}{\boxspacing}
\begin{Verbatim}[commandchars=\\\{\}]
\PY{n}{S} \PY{o}{=} \PY{p}{[}\PY{l+m+mi}{423}\PY{p}{,} \PY{l+m+mi}{779}\PY{p}{,} \PY{l+m+mi}{434}\PY{p}{,} \PY{l+m+mi}{371}\PY{p}{,} \PY{l+m+mi}{244}\PY{p}{,} \PY{l+m+mi}{245}\PY{p}{,} \PY{l+m+mi}{753}\PY{p}{,} \PY{l+m+mi}{519}\PY{p}{,} \PY{l+m+mi}{106}\PY{p}{,} \PY{l+m+mi}{167}\PY{p}{,} \PY{l+m+mi}{34}\PY{p}{,} \PY{l+m+mi}{650}\PY{p}{,}
     \PY{l+m+mi}{865}\PY{p}{,} \PY{l+m+mi}{605}\PY{p}{,} \PY{l+m+mi}{441}\PY{p}{,} \PY{l+m+mi}{190}\PY{p}{,} \PY{l+m+mi}{774}\PY{p}{,} \PY{l+m+mi}{512}\PY{p}{,} \PY{l+m+mi}{970}\PY{p}{,} \PY{l+m+mi}{394}\PY{p}{,} \PY{l+m+mi}{518}\PY{p}{,} \PY{l+m+mi}{887}\PY{p}{,} \PY{l+m+mi}{908}\PY{p}{,} \PY{l+m+mi}{971}\PY{p}{,} \PY{l+m+mi}{14}\PY{p}{]}

\PY{n}{result} \PY{o}{=} \PY{n}{integer\PYZus{}partition}\PY{p}{(}\PY{n}{S}\PY{p}{,} \PY{l+s+s1}{\PYZsq{}}\PY{l+s+s1}{ip\PYZus{}feasible.lp}\PY{l+s+s1}{\PYZsq{}}\PY{p}{)}

\PY{n+nb}{print}\PY{p}{(}\PY{l+s+s1}{\PYZsq{}}\PY{l+s+se}{\PYZbs{}n}\PY{l+s+s1}{Result partition is: }\PY{l+s+se}{\PYZbs{}n}\PY{l+s+s1}{\PYZsq{}}\PY{p}{)}
\PY{n}{print\PYZus{}solution}\PY{p}{(} \PY{n}{result} \PY{p}{)}
\end{Verbatim}
\end{tcolorbox}

    \begin{Verbatim}[commandchars=\\\{\}]
Gurobi Optimizer version 9.0.2 build v9.0.2rc0 (linux64)
Optimize a model with 26 rows, 50 columns and 100 nonzeros
Model fingerprint: 0x7190bcc4
Variable types: 0 continuous, 50 integer (50 binary)
Coefficient statistics:
  Matrix range     [1e+00, 1e+03]
  Objective range  [1e+01, 1e+03]
  Bounds range     [1e+00, 1e+00]
  RHS range        [1e+00, 1e+00]
Found heuristic solution: objective 1434.0000000
Presolve removed 26 rows and 50 columns
Presolve time: 0.00s
Presolve: All rows and columns removed

Explored 0 nodes (0 simplex iterations) in 0.03 seconds
Thread count was 1 (of 4 available processors)

Solution count 2: 0

Optimal solution found (tolerance 1.00e-04)
Best objective 0.000000000000e+00, best bound 0.000000000000e+00, gap 0.0000\%

Result partition is:

sum(A) = sum([779, 244, 190, 512, 970, 394, 518, 887, 908, 971, 14]) = 6387
sum(B) = sum([423, 434, 371, 245, 753, 519, 106, 167, 34, 650, 865, 605, 441,
774]) = 6387
    \end{Verbatim}

    \begin{tcolorbox}[breakable, size=fbox, boxrule=1pt, pad at break*=1mm,colback=cellbackground, colframe=cellborder]
\prompt{In}{incolor}{6}{\boxspacing}
\begin{Verbatim}[commandchars=\\\{\}]
\PY{n}{S} \PY{o}{=} \PY{p}{[}\PY{l+m+mi}{372}\PY{p}{,} \PY{l+m+mi}{734}\PY{p}{,} \PY{l+m+mi}{954}\PY{p}{,} \PY{l+m+mi}{124}\PY{p}{,} \PY{l+m+mi}{985}\PY{p}{,} \PY{l+m+mi}{759}\PY{p}{,} \PY{l+m+mi}{785}\PY{p}{,} \PY{l+m+mi}{462}\PY{p}{,} \PY{l+m+mi}{522}\PY{p}{,} \PY{l+m+mi}{70}\PY{p}{,} \PY{l+m+mi}{204}\PY{p}{,} 
     \PY{l+m+mi}{751}\PY{p}{,} \PY{l+m+mi}{343}\PY{p}{,} \PY{l+m+mi}{57}\PY{p}{,} \PY{l+m+mi}{152}\PY{p}{,} \PY{l+m+mi}{209}\PY{p}{,} \PY{l+m+mi}{724}\PY{p}{,} \PY{l+m+mi}{405}\PY{p}{,} \PY{l+m+mi}{867}\PY{p}{,} \PY{l+m+mi}{177}\PY{p}{,} \PY{l+m+mi}{701}\PY{p}{]}

\PY{n}{result} \PY{o}{=} \PY{n}{integer\PYZus{}partition}\PY{p}{(}\PY{n}{S}\PY{p}{,} \PY{l+s+s1}{\PYZsq{}}\PY{l+s+s1}{ip\PYZus{}infeasible.lp}\PY{l+s+s1}{\PYZsq{}}\PY{p}{)}

\PY{n+nb}{print}\PY{p}{(}\PY{l+s+s1}{\PYZsq{}}\PY{l+s+se}{\PYZbs{}n}\PY{l+s+s1}{Result partition is: }\PY{l+s+se}{\PYZbs{}n}\PY{l+s+s1}{\PYZsq{}}\PY{p}{)}
\PY{n}{print\PYZus{}solution}\PY{p}{(} \PY{n}{result} \PY{p}{)}
\end{Verbatim}
\end{tcolorbox}

    \begin{Verbatim}[commandchars=\\\{\}]
Gurobi Optimizer version 9.0.2 build v9.0.2rc0 (linux64)
Optimize a model with 22 rows, 42 columns and 84 nonzeros
Model fingerprint: 0x39dc88a7
Variable types: 0 continuous, 42 integer (42 binary)
Coefficient statistics:
  Matrix range     [1e+00, 1e+03]
  Objective range  [6e+01, 1e+03]
  Bounds range     [1e+00, 1e+00]
  RHS range        [1e+00, 1e+00]
Found heuristic solution: objective 3063.0000000
Presolve removed 22 rows and 42 columns
Presolve time: 0.00s
Presolve: All rows and columns removed

Explored 0 nodes (0 simplex iterations) in 0.03 seconds
Thread count was 1 (of 4 available processors)

Solution count 2: 1

Optimal solution found (tolerance 1.00e-04)
Best objective 1.000000000000e+00, best bound 1.000000000000e+00, gap 0.0000\%

Result partition is:

sum(A) = sum([759, 785, 343, 57, 152, 209, 724, 405, 867, 177, 701]) = 5179
sum(B) = sum([372, 734, 954, 124, 985, 462, 522, 70, 204, 751]) = 5178
    \end{Verbatim}

    The results of running the model on the two instances, show that our
implementation is capable of finding an exact solution when it exists.
And if it does not, it finds the best approximation as examplified by
the last instance.

    \section{References}\label{references}

{[}1{]} The kakuro puzzle, https://www.puzzle-kakuro.com/\\
{[}2{]} The Gurobi Optimizer,
https://www.gurobi.com/products/gurobi-optimizer/\\
{[}3{]} Source code and files repository,
https://github.com/faustind/gurobipy\\
{[}4{]} The Algorithm Design Manual, 2nd Ed, Section 13.10, Steven S. Skiena


    % Add a bibliography block to the postdoc
    
    
    
\end{document}
